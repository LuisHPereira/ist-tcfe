\section{Theoretical Analysis}
\label{sec:analysis}
\paragraph{}
\par In this section, the circuit shown in Figure \ref{circuit} is analyzed
theoretically. In the table below the known values are presented. The units are $kOhm$, $V$, $mS$, $\mu F$.
\begin{table}[H]
    \centering
    \begin{tabular}{|c|c|}
    \hline
        \input{../mat/data.tex}
    \end{tabular}
    \caption{Known Data}
    \label{data}
\end{table}
\subsection{For $t<0$}
\paragraph{}
\par According to the equation given in the description of the assignment, $v_s=V_s$ when $t<0$. In this situation the capacitor behaves like an open circuit, which means we can consider $I_c = 0$.
\par Using the known values and the deduction above, it is possible to solve the circuit, using the node method.
The following system is the one used to solve this circuit.
$$
\begin{cases} 
	V_1 = V_s \\ 
	G_1 (V_1-V_2) + G_2 (V_3-V_2) + G_3 (V_5-V_2) = 0 \\
	G_2 (V_2-V_3) + K_b (V_2-V_5) = 0 \\
	V_4 = 0 \\
	G_1 (V_2-V_1) + G_4 (V_5-V_4) + \frac{V_8-V_5}{K_d} = 0 \\ 
	G_5 (V_5-V_6) + K_b (V_5-V_2) = 0 \\
	G_6 (V_4-V_7) + \frac{V_8-V_5}{K_d} = 0 \\ 
	G_7 (V_7-V_8) + \frac{V_8-V_5}{K_d} = 0 
\label{system1}
\end{cases}
$$

With the aid of Octave to solve the system, the following values are obtained (table \ref{table1a} and \ref{table1b}).

\begin{table}[H]
    \begin{minipage}{.5\linewidth}
      \centering
        \begin{tabular}{|c|c|}
    	\hline
        	\input{../mat/1voltages.tex}
        \end{tabular}
        \caption{Voltages in all nodes in V}
        \label{table1a}
    \end{minipage}%
    \begin{minipage}{.5\linewidth}
      \centering
        \begin{tabular}{|c|c|}
   	\hline
        	\input{../mat/1currents.tex}
        \end{tabular}
        \caption{Currents in all branches in mA}
        \label{table1b}
    \end{minipage} 
\end{table}
 

\subsection{For $t=0$}
\paragraph{}

\par In this situation, the capacitor is replaced by a voltage source $Vx$, that imposes a voltage of the same value of that that was determined in the previous situation for the capacitor. It is then possible to reach a value for $R_eq$ as seen from the capacitor because it is now possible to solve a new system of equations (\ref{system 2}) in which $vs=0$ that allows the determination of $I_x$, this being the current passing through the capacitor.
\par Translating the system below into a matrix, it is possible to determine $R_{eq}$.
\par $R_{eq}$ is necessary because it enables the determination of the natural solution of the system through the $RC$ constant. This application will be demonstrated in the next exercise.

$$
\begin{cases} 
	V_1 = 0 \\ 
	G_1 (V_1-V_2) + G_2 (V_3-V_2) + G_3 (V_5-V_2) = 0 \\
	G_2 (V_2-V_3) + K_b (V_2-V_5) = 0 \\
	V_4 = 0 \\
	G_1 (V_2-V_1) + G_4 (V_5-V_4) + \frac{V_8-V_5}{K_d} = 0 \\ 
	G_5 (V_5-V_6) + K_b (V_5-V_2) + \frac{R_{eq}}{V_x} = 0 \\
	G_6 (V_4-V_7) + \frac{V_8-V_5}{K_d} = 0 \\ 
	G_7 (V_7-V_8) + \frac{V_8-V_5}{K_d} = 0 \\
	V_6 - V_8 = V_x
\label{system 2}
\end{cases}
$$



Using the values computed by Octave once again, tables \ref{table2a} and \ref{table2b} are built.

\begin{table}[H]
    \begin{minipage}{.5\linewidth}
      \centering
        \begin{tabular}{|c|c|}
    	\hline
        	\input{../mat/2voltages.tex}
        \end{tabular}
        \caption{Voltages in all nodes in V}
        \label{table2a}
    \end{minipage}%
    \begin{minipage}{.5\linewidth}
      \centering
        \begin{tabular}{|c|c|}
   	\hline
        	\input{../mat/2currents.tex}
        \end{tabular}
        \caption{Currents in all branches in mA}
        \label{table2b}
    \end{minipage} 
\end{table}


\par The value for $V_x$ was estabilished as $V_6 - V_8$ being $V_6$ and $V_8$ determined in the subsection above.

\subsection{Natural Solution}
\paragraph{}

\par The objective for this part of the assignment was to determine the natural solution of $V_6$. Using the information calculated in the previous subsection, $V_x$ specifically, it is possible with the use of \ref{equationnat} to achieve the goal, given this is a RC circuit, as shown in figure \ref{3}. 

\begin{equation}
	V_{6n} = A e^{\frac{-t}{\tau}}  \ \ \  \ \ \ \ \tau = R_{eq} C
	\label{equationnat}
\end{equation}



\begin{figure}[H]
    \includegraphics[width=0.5\linewidth]{3.eps}
    \centering
    \caption{Natural Solution}
    \label{3}
\end{figure}

\subsection{Forced Solution}
\paragraph{}

\par In this sector, it is now necessary to determine the forced solution for $V_{6}f(t)$, with a frequency of 1000Hz.Utilizing the suggestion, it is introduced a phasor with a constant $V_s$ of 1V.
It was than ran a similar node analysis to the previous, using $V_s$ as the voltage source, while replacing the capacitance $C$ of the capacitor with its impendance $Z$. The following equations were then deducted: 

\begin{equation}
	Z_c = \frac{1}{\omega C j}     \ \ \  \ \ \ \ \  \ \ \ \ \ \ \ \omega = 2 \pi f
\end{equation}


$$
\begin{cases} 
	V_1 = j \\ 
	G_1 (V_1-V_2) + G_2 (V_3-V_2) + G_3 (V_5-V_2) = 0 \\
	G_2 (V_2-V_3) + K_b (V_2-V_5) = 0 \\
	V_4 = 0 \\
	G_1 (V_2-V_1) + G_4 (V_5-V_4) + \frac{V_8-V_5}{K_d} = 0 \\ 
	G_5 (V_5-V_6) + K_b (V_5-V_2) + \frac{V_8-V_5}{Z_c} = 0 \\
	G_6 (V_4-V_7) + \frac{V_8-V_5}{K_d} = 0 \\ 
	G_7 (V_7-V_8) + \frac{V_8-V_5}{K_d} = 0 
\label{system 4}
\end{cases}
$$
\par Using the following equation:

\begin{equation}
	V_{complex_i} = V_i e^{-j phase(i)}
\end{equation}

\par Now, with resort to angle and abs commands in Octave, the tables below were made:

\begin{table}[H]
    \begin{minipage}{.5\linewidth}
      \centering
        \begin{tabular}{|c|c|}
    	\hline
        	\input{../mat/4absolute.tex}
        \end{tabular}
        \caption{Complex Amplitudes}
        \label{table4a}
    \end{minipage}%
    \begin{minipage}{.5\linewidth}
      \centering
        \begin{tabular}{|c|c|}
   	\hline
        	\input{../mat/4phase.tex}
        \end{tabular}
        \caption{Phases}
        \label{table4b}
    \end{minipage} 
\end{table}

It is possible, therefore, to conclude that:

\begin{equation}
	V_{6f} = 5.662518e-01 e^{-1.423112 j}
\end{equation}

\subsection{Natural and Forced Superimposed}
\paragraph{}

\par In this section, it is necessary to convert the phasor into real time functions, in order to find a function to evaluate $V_6$ for $t>0$, which was achieved by adding the natural and forced solutions, for a frequency of 1000Hz. The equation for $V_s(t)$ for $t>0$ was already on the initial circuit diagram. The two equations are:


\begin{equation}
	V_{ifinal} = V_{in} + V_{if}
\end{equation}

So, calculating $V_{6final}$:
\begin{equation}
	V_{6}(t) = e^{\frac{-t}{R_{eq} C}} + A e^{-j phase(i)}
\end{equation}

\begin{figure}[H]
    \includegraphics[width=0.5\linewidth]{5.eps}
    \centering
    \caption{Natural and Forced Superimposed}
    \label{pha}
\end{figure}




\subsection{Frequency Response}
\paragraph{}

\par The main focus of this procedure is to determine the frequency response $v_6(f)$, $v_c(f)$ and $v_s(f)$ for a range from 0.1Hz to 1Mhz (in a logarithmic scale)
\par The graphs for phase and magnitude were then plotted, with the magnitude (in dB) being calculated with the aid of the Octave abs function. The phase is calculated with the angle function of Octave, but not without being converted from radians to degrees.

\begin{equation}
	Z=\frac{1}{2 \pi f C j}
\end{equation}

with $f$ being a logharitmic scale vector from -1 to 6 with 100 entries.

The system of equations used was:

$$
\begin{cases} 
	V_1 = j \\ 
	G_1 (V_1-V_2) + G_2 (V_3-V_2) + G_3 (V_5-V_2) = 0 \\
	G_2 (V_2-V_3) + K_b (V_2-V_5) = 0 \\
	V_4 = 0 \\
	G_1 (V_2-V_1) + G_4 (V_5-V_4) + \frac{V_8-V_5}{K_d} = 0 \\ 
	G_5 (V_5-V_6) + K_b (V_5-V_2) + \frac{V_8-V_5}{Z_c} = 0 \\
	G_6 (V_4-V_7) + \frac{V_8-V_5}{K_d} = 0 \\ 
	G_7 (V_7-V_8) + \frac{V_8-V_5}{K_d} = 0 
\label{system 6}
\end{cases}
$$

The $V_1$, $V_6 - V_8$ and $V_6$ calculated were then assigned to $Vsfre(k)$, $Vxfre(k)$ and $V6fre(k)$, respectively 

Lastly, the two following graphics were plotted using a base 10 logarithmic scale for frequencies, the logarithmic value of the $abs$ of the variables stated above and the angle of these complex variables, converted to degrees.



\begin{figure}[H]
    \includegraphics[width=0.495\linewidth]{6a.eps}
    \hfill\includegraphics[width=0.495\linewidth]{6b.eps}
    \centering
    \caption{Magnitude (dB) - Frequency (Hz) / Phase (Degrees) - Frequency (Hz)}
    \label{mag}
\end{figure}

