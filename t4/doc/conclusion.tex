\section{Conclusion}
\label{sec:conclusion}
\paragraph{}
\par The first element of this conclusion will be the comparison between theoretical results and the ones obtained using NGSpice. The first comparison will be between the voltages in the collector, base and emitter of the transistor and the currents in each terminal from the Operating Point analysis: 

\begin{table}[H]
	\begin{minipage}{.5\linewidth}
		\centering
		\begin{tabular}{|c|c|}
		\hline
		\input{../mat/TOTAL.tex}
		\end{tabular}
		\caption{Theoretical Analysis}
		\label{table1a}
	\end{minipage}
	\begin{minipage}{.5\linewidth}
		\centering
		\begin{tabular}{|c|c|}
		\hline
		\input{../sim/zin_TAB.tex}
		\input{../sim/zout_TAB.tex}
		\input{../sim/sim_TAB.tex}
	\end{tabular}
		\caption{Simulation Analysis}
		\label{table1b}
	\end{minipage} 
\end{table}

\par The second element to the conclusion is the analysis of the project as a whole and final comments:
\par The main goal of this laboratory assignment was to project an audio amplifier that allowed for maximum gain of voltage, spending the least amount of resources possible. Such objective was fulfilled as the results obtained were the ones to be expected, as the voltage gain is quite significant, despite the cost also being quite significant.
\par Nevertheless, despite our effort to obtain similar results through the two methods: theoretical calculations and simulations, such was not attainable, mainly due to the non-linearity of the transistors which precluded us from achieving so.
\par In the end, analyzing the cost-merit relationship through the simulations, one concludes that these are results which ultimately satisfy our intention.






