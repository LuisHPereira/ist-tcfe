\section{Introduction}
\label{sec:introduction}
\paragraph{}
\par The goal of this assignment is to design an audio amplifier circuit and, similar to previous projects, analyze it both theoretically and through simulation, resorting to NGSpice. This circuit configuration has the purpose of amplifying an audio signal received and outputting it into a speaker. In this specific case, the input received will be of 10mV and it will output into a 8 $\Omega$ speaker.
\par The path taken by the signal starts in a gain stage, comprised mainly of a NPN transistor, which will amplify it while also increasing its impedance. In order to reduce said impedance, the signal goes through an output stage, which is composed mainly by a PNP transistor, keeping the same amplitude on average, making it suitable for outputting into the speaker without any significant gain loss.
\par With this information, a gain and output stage were designed, creating the total circuit seen in the following figure:
\begin{figure}[H]
    \includegraphics[width=0.5\linewidth]{Lab4.pdf}
    \centering
    \caption{Studied Circuit}
    \label{circuit}
\end{figure}
\par In the following table, the values for the components are presented, with the units being V, $\Omega$ and F.

\begin{table}[H]
    \centering
    \begin{tabular}{|c|c|}
    \hline
        \input{../mat/IV.tex}
    \end{tabular}
    \caption{Initial Values}
    \label{table4a}
\end{table}

\par The quality of the amplifier will be dictated by the expression below:
\[Merit=\frac{Gain \cdot Bandwidth}{Cost \cdot LowerCutOffFrequency}\]


In Section~\ref{sec:analysis}, a theoretical analysis of the circuit is
presented. In Section~\ref{sec:simulation}, the circuit is analysed by
simulation using NGSpice, with its results being compared to the theoretical results obtained in
Section~\ref{sec:analysis} in the Section~\ref{sec:conclusion}, while also outlining in this section the conclusions of this study.	
