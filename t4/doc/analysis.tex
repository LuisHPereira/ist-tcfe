\section{Theoretical Analysis}
\label{sec:analysis}
\paragraph{}

\par In the Introduction section, it was explained how the circuit was divided into two stages: a Gain Stage and an Output stage. Resorting to this information, it is possible to deduct a Theoretical Model, to which we can apply the Operating Point analysis in order to obtain the values for gain, input and output voltages.
\par This was made possible by the equations given by the professor and the information transmitted through the lectures.
\par The values desired for the Gain stage are presented in the table below:


\begin{table}[H]
    \centering
    \begin{tabular}{|c|c|}
    \hline
        \input{../mat/GS.tex}
    \end{tabular}
    \caption{Gain Stage}
    \label{table4a}
\end{table}
\par The same values but for the output stage are shown in the following table:
\begin{table}[H]
    \centering
    \begin{tabular}{|c|c|}
    \hline
        \input{../mat/OS.tex}
    \end{tabular}
    \caption{Output Stage}
    \label{table4a}
\end{table}

\par The values for gain and input and output impedances for the full circuit is shown in the following table:

\begin{table}[H]
    \centering
    \begin{tabular}{|c|c|}
    \hline
        \input{../mat/TOTAL.tex}
    \end{tabular}
    \caption{Full circuit}
    \label{table4a}
\end{table}



\par A plot for the frequency response can then be made, as one can observe in the figure below:

\begin{figure}[H]
	\includegraphics[width=0.5\linewidth]{gain.eps}
	\centering
	\caption{Gain plot - $\frac{V_o(f)}{V_i(f)}$}
	\label{pha}
\end{figure}


