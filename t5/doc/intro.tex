\section{Introduction}
\label{sec:introduction}
\paragraph{}
\par This lab assignment has the purpose of designing and implementing a BandPass Filter (BPF) into an OpAmp, short for Operational Amplifier, with 1 kHz of central frequency and a gain on said frequency of 40dB. To calculate the quality of the designed filter, the following Merit expression is used:
\[Merit=\frac{1}{Cost \cdot (GainDeviation + CentralFrequecyDeviation + 10^{-6})}\]
\par The circuit designed is presented below followed by a table containing the value for each component in units of Volts (V), Ohms ($\Omega$) or Farads (F).
            
\begin{figure}[H]
    \includegraphics[width=0.5\linewidth]{Lab5.pdf}
    \centering
    \caption{Studied Circuit}
    \label{circuit}
\end{figure}
\par In the following table, the values for the components are presented, with the units being V, $\Omega$ and F.

\begin{table}[H]
    \centering
    \begin{tabular}{|c|c|}
    \hline
        \input{../mat/datatab.tex}
    \end{tabular}
    \caption{Initial Values}
    \label{table4a}
\end{table}

\par In Section~\ref{sec:analysis}, a theoretical analysis of the circuit is
presented. In Section~\ref{sec:simulation}, the circuit is analysed by
simulation using NGSpice, with its results being compared to the theoretical results obtained in
Section~\ref{sec:analysis} in the Section~\ref{sec:conclusion}, while also outlining in this section the conclusions of this study.	
